\documentclass{article} % For LaTeX2e
\usepackage{nips15submit_e,times}
\usepackage{hyperref}
\usepackage{url}
\usepackage{times}
\usepackage{epsfig}
\usepackage{graphicx}
\usepackage{amsmath}
\usepackage{amssymb}
\usepackage{multirow}

\usepackage{algorithm}
\usepackage{algpseudocode}
\usepackage{amsthm}
%\documentstyle[nips14submit_09,times,art10]{article} % For LaTeX 2.09
\theoremstyle{definition}
\newtheorem{mydef}{Definition}
\newcommand{\CXNote}[1]{\textbf{[Stephen: {#1}]}}
\newcommand{\HHNote}[1]{\textbf{[hhe: {#1}]}}


\title{The Imitation Game:\\ Object Detection in 20 Questions}


\author{
Xi (Stephen) Chen\\
Department of Computer Science\\
University of Maryland\\
College Park, MD 20742 \\
\texttt{chenxi@umiacs.umd.edu} \\
\AND
He He \\
Department of Computer Science\\
University of Maryland\\
College Park, MD 20742 \\
\texttt{hhe@cs.umd.edu} \\
\And
Gregory Shakhnarovich\\
Toyota Technological Institute at Chicago \\
Chicago, IL\\
\texttt{gregory@ttic.edu} \\
\And
Larry S. Davis \\
Department of Computer Science\\
University of Maryland\\
College Park, MD 20742 \\
\texttt{lsd@umiacs.umd.edu} 
}

% The \author macro works with any number of authors. There are two commands
% used to separate the names and addresses of multiple authors: \And and \AND.
%
% Using \And between authors leaves it to \LaTeX{} to determine where to break
% the lines. Using \AND forces a linebreak at that point. So, if \LaTeX{}
% puts 3 of 4 authors names on the first line, and the last on the second
% line, try using \AND instead of \And before the third author name.

\newcommand{\fix}{\marginpar{FIX}}
\newcommand{\new}{\marginpar{NEW}}

%\nipsfinalcopy % Uncomment for camera-ready version

\begin{document}

\maketitle

\begin{abstract}
We propose a new strategy for simultaneous object detection and segmentation. Instead of evaluating all object detectors at all possible locations in the
image, we develop a divide-and-conquer approach by sequentially posing questions about the query and its related context---like playing a ``Twenty Questions'' game---to decide where to search for the object. We formulate the problem as a Markov Decision Process and learn a search policy by imitation learning. At each step, the policy dynamically selects a question based on the query, the scene and observed responses given by object detectors evaluated so far.
Experimental results show that our
algorithm reduces over 30\% of the object proposal evaluation with little loss in average precision compared to exhaustive search. 
Our learned policy also achieves better speed-accuracy tradeoff than the random search strategy.
\end{abstract}

%%%%%%%%% BODY TEXT
\section{Introduction}
%We employ a search and prune in the graph of state space to learn action classifiers for each query class. At test time given the current state (previous detector responses and the current image) representing by the CNN features of the masked search space, the action classifier will select the most highly scored action. We also incorporate early rejection as an action which will reject further detection if the it isn't likely to contain the query object in the current masks. 

Object detection and segmentation in complex scenes is a central and challenging problem in computer vision and robotics.
%Given an image, for example, Figure~\ref{fig:20Qintro}, our goal is to answer the query: is there a car in the scene, and if yes, to locate it with a bounding box or pixel-wise labels.
This problem is usually tackled by running multiple object detectors exhaustively on densely sampled sliding windows~\cite{felzenszwalb2010object} or category-independent object proposals~\cite{carreira2012cpmc,van2011segmentation,arbelaez2014multiscale}. 
Such methods are time-consuming since they need to evaluate a large number of object hypotheses, and can easily introduce false positives if exclusively considering local appearance.
% In addition, due to variations in data distribution, occlusion and viewpoint change, object models may not always capture the appearance of objects and ambiguity arises. 
%In the example of Figure~\ref{fig:20Qintro}, since the viewpoint and the scale of the cars are not similar to those in common training images, it is difficult for the car detector to recognize and locate them.

Instead of checking all hypotheses indiscriminately and exhaustively, humans only look for a set of related objects in a given context~\cite{biederman1982scene, hock1974contextual}. Context information is an effective cue for humans to detect low-resolution or small objects in cluttered scenes~\cite{parikh2012exploring}. Many contextual models have been proposed to capture relationships between objects at the semantic level to reduce ambiguities from
unreliable independent detection results~\cite{gould2009decomposing, galleguillos2010context, ladicky2010graph}. %For example, because roads and buildings often co-occur with cars, knowing the existence of these objects can help us infer the locations of cars.  
However, such methods still need to evaluate the high order co-occurrence statistics and spatial relations of the query object with \emph{all} other object classes in the scene, some of which may not be informative and even introduce unnecessary confusion.  

By contrast, humans do not process the whole scene at once: it is an active process that sequentially samples the optic array in an intelligent, task-specific way~\cite{najemnik2005optimal}. Research in neuro-science has revealed that when humans search for a target, those objects that are associated to the query will reinforce attention with the query and weaken recognition of unrelated distractions~\cite{moores2003associative}. 
%This is highly inefficient since many non-informative contextual objects have to be queried. 
For instance, in Figure~\ref{fig:20Qintro}, when we search for cars, knowing the top of the scene is sky does not help distinguish whether there is a car or a boat since both can be under the sky; 
on the other hand, observing a road instead of water in the lower part gives a strong indication of the existence of cars. 
Additionally, we know from experience that roads, cars and buildings are likely to co-occur and cars are often found beside buildings. 
Therefore to find cars, humans tend to first look for roads and then search around buildings, instead of looking up for the sky. %And if we know there is road, we do not need to ask about water.
%We note that the set of related object classes and the order of asking questions about them is dynamic given a specific query in the scene and knowledge of previous observations.
This motivates us to raise the question: \textit{can object detection algorithms decide where to look for objects of a query class more efficiently and accurately by exploring a few related context cues dynamically, similar to humans?}

We formulate the object detection problem as a Markov Decision Process (MDP). 
We use imitation learning to learn a context-driven policy that sequentially and dynamically selects the most informative context class to explore based on past observations, and gradually refine the search area for the query class. 
We show our framework in Figure~\ref{fig:flowchart}.  Specifically, like playing a Twenty Questions game, at each step the policy asks for information about a context class based on the query and responses from previous contextual classifiers. 
We then run the detector/classifier of the inquired context class. Based on the responses, we further refine the search area for the query class using spatial-aware contextual models. 
This process of context querying and search area refining is repeated until the policy thinks enough contextual information has been gathered and decides to stop. 
Finally, we run the query object detector in the refined search area and output the result. 
Besides asking for contextual information, our policy can reject a query early to avoid unnecessary detection if it determines that there is a small chance of having the query object in the scene. 
The decision of early rejection can be made even before running any object detector, therefore we can eliminate a large amount of unnecessary computation. 

Object detection experimental results on images of complex scenes (the PASCAL VOC 2010 dataset) show that our algorithm produces a search area that has better overlap with the target object by leveraging its context, 
thus significantly reduces the number of object proposals to consider and detectors to evaluate\HHNote{Add the number here}.
Although with less computation, our method achieves mean average precision (mAP) comparable to the exhaustive search method. 
To the best of our knowledge, this is one of the first few approaches that solve the challenging task of simultaneous object detection and segmentation in complex real scenes by applying imitation learning to learn a policy fully driven by semantic context.


\begin{figure*}[htb]
\begin{center}
\includegraphics[width=\linewidth]{figures/iccv20q-overview.pdf}
\caption{Illustration of our sequential search for objects in 20 context driven questions.}
\label{fig:20Qintro}
\end{center}
\end{figure*}

% The main contributions of the paper are:
% \HHNote{seems contribution 1 and 2 can be combined, they are both novelty of a new learning algorithm to the problem. e.g., we formulate and learn...}
% \begin{itemize}
% \item a novel formulation of the object detection problem as a Markov Decision Process and a dynamic, closed-loop policy learned by imitation learning to decide which detectors to run next and where to look for the query object iteratively
% \item a general and unified probabilistic framework incorporating responses from multi-class object detectors and contextual classifiers to update the search area for the target
% \item a data-driven context model that not only encodes co-occurrence but also spatial relations by efficient weighted vote maps from exemplars\HHNote{This is not explained before and it's hard to understand here what the context model is}.
% \end{itemize}





\section{Related Work}
\label{sec:relatedwork}

{\bf{Sequential Testing}}. 
The ``20 question'' approach to pattern recognition dates back to Blanchard and Geman~\cite{blanchard2005hierarchical}, motivated by the large number of possible explanations in scene interpretation. They formally studied coarse-to-fine search in the theoretical framework of sequential hypothesis testing, and proposed optimal strategies considering both the cost and effectiveness of each test. Although they did not consider contextual information, their work provides a theoretical foundation for the design of sequential algorithms.

There are several works~\cite{gao2011active} on classifying objects by running classifiers sequentially in an active order.~\cite{branson2010visual} proposed an information gain based approach to iteratively pose questions for users and incorporate human responses and computer vision detector results for fine-grained classification.
~\cite{sergey2012timely} formulated object classification as a Markov decision process, 
where actions are the detector to deploy next. 
The model maintains a belief of object classes and keeps updating it based on new observations.  
They used reinforcement learning to train the detector selection policy, 
which becomes expensive when the number of classes and data size is large due to exploration. However, these approaches only focus on classifying objects. They have not addressed the challenging problem of simultaneous segmentation and localization of objects in a multi-class scene as we do in this chapter, and did not exploint inter-object context.

~\cite{bogdan2012context} applied a sequential decision making framework to window selection. The next window is selected based on votes of previously evaluated windows. However, the voting process needs to look up nearest neighbors in hundreds of thousands of exemplar window pairs in the training set because their context is at the exemplar/instance level, which is highly inefficient. In contrast, our context modeling is semantically aware so we do not compute nearest neighbors over hundreds of thousands of windows in a high dimensional descriptor space to retrieve the voters, we only need votes from a few regions within the search space of context class instead of sampling hundreds of windows in~\cite{bogdan2012context}. Our context model achieves good accuracy while greatly reducing computational complexity.

{\bf Object Detection}. 
A common approach to object detection is based on applying gradient based features over densely sampled sliding windows~\cite{felzenszwalb2010object}.Such methods achieve good results on classes like human and vehicles, but they are very inefficient since they evaluate thousands of windows in an image, and false positve detections arise. To reduce the number of windows evaluated.~\cite{lampert2009efficient} proposed a subwindow search based on a branch-and-bound scheme and only evaluates the high scoring windows. Recently, category independent object proposals~\cite{carreira2012cpmc,van2011segmentation,arbelaez2014multiscale} have been proposed to generate a small number of high quality regions or windows that are likely to be objects. These approaches dramatically reduce the number of candidates and reduce false positive detections. Using these object proposals~\cite{girshick14CVPR, BharathECCV2014} train and apply deep neural network models on large datasets to learn the feature extractor and classifiers, and achieve state-of-the-art performance on the Pascal VOC detection challenge. 

{\bf Object Recognition using Context}. 
Context has been shown to improve object recognition and detection. Model-based approaches learn the appearance of semantic categories and relations among them
using a parametric model. In~\cite{gould2009decomposing, galleguillos2010context,mottaghirole, shotton2006textonboost, ladicky2010graph}, CRF models are used to combine unary potentials based on visual features extracted from superpixels with neighborhood constraints and low level context. Inter-object context in the scene has also been shown to improve recognition~\cite{galleguillos2010context, chen2011piecing}. Most of these context models are used as post-detection smoothing after all classifers are run as unary potentials, and then they are jointly incorporated in inference regardless of their importance to different kinds of objects and scenes. Our framework, in contrast, evaluates the informativeness of context in an active loop before classifications of all objects are made, and goes beyond simple co-occurence statistics.

\section{Approach}
\HHNote{This paragraph needs a better structure. Also, some concepts are not defined.}

Our framework is shown in Figure~\ref{fig:flowchart}\HHNote{Mention the figure when you plan to explain it. Maybe after the high level description of the approach.}. Our goal is to learn \HHNote{policy is not defined yet. Maybe have a background subsection for MDP first? or don't mention policy at all, just say we want to dynamically ...}a policy $\pi(s)$ that can dynamically determine a sequence \HHNote{``determine a sequence'' is confusing because it does not predict a ``sequence''. Consider: sequentially select a contextual object class to detect ...}of contextual \HHNote{object class?} classes to detect to help \HHNote{help whom? I'd just say to narrow
down}narrow down the search area of the query or
make an early rejection, under certain budget constraint\HHNote{Too long, hard to read. Perhaps break into: ... object class under budget constraints, which narrows down...}. We model it in a reinforcement learning framework by introducing the Markov Decision Process (MDP) \HHNote{Model the problem as a MDP and use RL to solve it}, which defines a single \textit{episode} \HHNote{not sure what episode means hear..}of selecting actions for the input image $X$ and target query class
$c_q$ \HHNote{$X$ and $c_q$ are not used here and they will be defined in the next section. Consider removing the notations}.

\begin{figure*}[htb]
\begin{center}
\includegraphics[width=\linewidth]{figures/flowchart_Q.pdf}
\caption{Framework of our context driven object searching. We first generate region hypotheses using object proposal algorithms, then the policy evaluates the current state and iteratively selects the action maximizing the Q-value function. Afterwards, the possible search locations are updated and the posterior probabilities of each category are evaluated for the next state.}
\label{fig:flowchart}
\end{center}

\end{figure*}

\subsection{Context Driven Object Search as Markov Decision Process (MDP)}
\label{sec:policy}

Specifically, given an image $X$ and a query object class $c_q$ in classes ${1,..,C}$, we define the MDP as follows.
\begin{mydef}
 The \textbf{detection action selection MDP} is defined by the tuple $(\mathcal{S}, \mathcal{A}, T(.), R(.), \gamma)$:
\begin{itemize}
\item \textbf{State} \HHNote{I would give some explanation of what state is before the math, e.g. a state contains necessary information for decision making}$s(t) = (X, R^t)\in \mathcal{S}$ that includes the image $X$ and the observations $R^t= \{r_1, r_2, ....,r_t\}$ over time till $t$.
\item The set of \textbf{actions} \HHNote{Again, some explanation: actions to be taken in each state}$\mathcal{A} = \{a_1, ..., a_C, \mbox{Stop}, \mbox{Reject}\} $, where $a_i$ is to detect class $c_i$\HHNote{Be more precise: what does detect mean: run the detector of class $c_i$ only? also give proposals?}, \textit{Reject} to reject \HHNote{if the process terminates after Reject, be explicit about it}query class in the image, and \textit{Stop} to output the search area and run query detector.
\item \textbf{State transition} function $T(s'|s,a, X)$ \HHNote{That's it? Need some explanation}
\item The \textbf{reward} function $R(s,a,s') \rightarrow \mathbb{R}$. \HHNote{That's it? Need some explanation}
\item The \textbf{discount} constant $\gamma$ defines a tradeoff between taking the action by greedily maximizing the immediate reward or the considering the long term expected reward.
\end{itemize}
\end{mydef}

%\newtheorem{def}[thm]{Definition}
% \begin{def}
% 
% \end{def}

\HHNote{Policy is not defined. Add something like: a policy (math) is a mapping from a state to an action. The goal of an MDP is to find a policy that maxmizes ...}

During test time, our search policy $\pi(s): \mathcal{S} \rightarrow \mathcal{A}$ iteratively selects an action $a^t$ in the action space $\mathcal{A}$ \HHNote{Consider $a^t \in \mathcal{A}$}. %= \{a_1, ..., a_C, Stop, Reject\} $, . 
Then the policy obtains response $r_t$ at time step $t$ given by the detection or classification results of action $a_t$. 

\HHNote{The causal relation ``since ... we want ... then define a policy'' here doesn't seem to be right. Policy and Q-value are defined by the MDP.} 
Formally, since we would like to select the actions dynamically, we want to learn a value function taking action $a$ in state $s$ under the policy $\pi$, denoted $Q^\pi(s,a): S\times A \rightarrow \mathbb{R}$, where $S$ is the space of all possible states, to assign a value to a potential action $a\in A$ given the current state. We can then define policy $\pi$ to take the action that maximizes the expected value:

\begin{eqnarray}
\label{eq:pi}
\pi(s) = \arg\max_{a_i\in A\backslash R} Q(s,a_i)
\end{eqnarray}

\subsection{Reward Function}

We define the \textit{immediate reward} $R$ as the immediate gain in intersection/union of the search space after conducting action $a_i$ at time step $t$ under state $s$ as:

\begin{eqnarray}
\label{eq:imreward}
R(s^t,a_i) =  \frac{X^{t+1}_i \cap X_q}{X^{t+1}_i \cup X_q} - \frac{X^{t}_i \cap X_q}{X^{t}_i \cup X_q}
\end{eqnarray}
where $X^{t+1}_i$ is the updated search area after executing action $a_i$ in state $s_t$, determined by the context models described in Section~\ref{sec:context}. $X_q$ is the groundtruth mask of the query object instances in the image. 
  
%at each time step $t$, we select a question $q_t$ and take action $a_t$ to evaluate it. Let $R^t = \{r_1, r_2, ....,r_t\}$ be the observations of responses to the actions taken at time $1...t$, where the response $r_t = p(c_t|X)$ is the detection or classification probability of class $c_t$ corresponding to question $q_t$. 




\subsection{Learning Context Driven Search Policy}
We propose a policy to use maximum expected reward to select $a_t$. To learn such a policy, we adopt a standard Q-learning algorithm~\cite{barto1998reinforcement}, where the action-value function is estimated by the Bellman equation recursively:
\begin{eqnarray}
\label{eq:bellman}
Q^*(s,a) = \mathbb{E}_{s'\sim \xi} \big[ R + \gamma \max_{a'} Q^*(s',a')|s,a \big]
\end{eqnarray}
where $a'$ and $s'\in\xi$ is the possible action-state pairs at the next time steps $t+1$.

Since the space of possible states $S$ is intractable \HHNote{The state space is continuous}, we use a linear function of the features of the states to approximate the $Q$-values \HHNote{linear approximation of the Q-values}:
\begin{eqnarray}
\label{eq:qvalue}
Q^{\pi}(s,a) = \theta_\pi^T \phi(s,a) 
\end{eqnarray}

where $\phi(s,a) = \phi((X, R^t),a) = \phi(X^t,a)$ is the feature of the undetermined area $X^t$ at time $t$ after observing detector responses of $a_1,...,a_t$. 

The parameters $\theta_\pi$ are learned by \textit{policy iteration}. We collect $(s,a,r,s')$ samples by running episodes starting from a random or empty state, then we search and prune in the tree of states and collect states samples  and corresponding features. An SVM regression is trained for each class to predict the Q-values given current states.

\subsection{Context Modeling}
\label{sec:context}
Since our task is not only to detect the object but also refine the search space of the query in the image as accurately as possible, conventional modeling of context as simple co-occurrence statistics is inadequate. Instead we present a data-driven location aware approach to represent the spatial correlation between the objects and the scene. 

Here we formulate the context $p(c_t|c,X)$ as a posterior of the probabilistic vote map $p(c|c_t,X_s)$ defined on each pixel $(x_i,x_j)\in X$ over the image, and the responses of class $c_t$ after action $a_t$:
\begin{eqnarray}
p(c_t|c,X) = \sum_{s\in X^t} p(c|c_t,X_s)p(c_t|X_s)
\end{eqnarray}

Given a refined search space $X^t\in X$ of a context class $c_t$ at time $t$, we formalize $p(c|c_t,X)$ as a weighted vote from the cooccurring region pairs of class $c_t$ and $c$ in training scenes. Let $(s_{c_t}^i, s_c^i)$ be the $i$-th pair of co-occurring regions of class $c_t$ and $c$, and $b_{c_t}^i$ and $b_c^i$ be the corresponding bounding boxes. We can now define the probabilistic vote map $p(c|c_t,X)$ as:
\begin{eqnarray}
\label{eq:votemap}
p(c|c_t,X_s)_{s\in X^t} = \frac{1}{Z_c}\sum_i W(s_{c_t}^i,s;\theta^W).T(b_{c_t}^i,b_c^i)
\end{eqnarray}
where $s\in X^t$ is a region within the search space of the context class $c_t$. $Z_c$ is the normalization function. $W(.)$ is a kernel measuring similarity of region $s$ with a training region $s_i$. $T(b_{c_t}^i,b_c^i)$ models the transformation from $b_{c_t}^i$ to $b_c^i$, including translation and scaling. Figure~\ref{fig:votemap} shows a few examples of the vote maps. We can see that with the exemplar based and semantically aware voting, the resulted vote maps give more accurate search area of the query objects.


\begin{figure}[ht!]
\begin{center}
\includegraphics[width=\linewidth]{figures/vote_sky_boat.pdf}
\end{center}
\caption{Examples of our weighted vote map for the context from sky to boat. The first rows are the training sample pairs of sky and boat and the second row is the test image and the resulted weighted voting map. The widths of the arrows denote the weighted similarity $W(s_{c_t}^i,s;\theta^W)$ between the test segment of sky (highlighted in yellow) and a training instance of sky segment (in light blue)}
\label{fig:vote_sky_boat}
\end{figure}


\begin{figure}[ht!]
\begin{center}
\includegraphics[width=\linewidth]{figures/votemap.pdf}
\end{center}
\caption{Examples of our context vote maps. Each pair of images corresponds to the original image and the vote-based probability map of object location from observed context. From (a) - (d) are the vote maps from water to boat, sky to boat, road to car and grass to cow, respectively. Best viewed in color.}
\label{fig:votemap}
\end{figure}



The final context probabilistic vote map is given by
\begin{eqnarray}
p(c_t|c,X) = \sum_{s\in X^t} p(c_t|X_s)\sum_i W(s_{c_t}^i,s;\theta^W).T(b_{c_t}^i,b_c^i)\nonumber\\
\end{eqnarray}
where $p(c_t|X_s)$ is the probabilities of $s$ as class $c_t$ after taking the action $a_t$ to run classification at time $t$.

\section{Implementation Details}
\subsection{Object Proposals}
We use MCG object proposals in~\cite{arbelaez2014multiscale} as object candidates. Since the object proposals mainly covers the objects,  we also generate a small number (20$\sim$30 per image) of segments using the stable segmentation algorithm from~\cite{chen2011piecing} to cover the whole scene including contextual classes. To reduce the computational overhead, our context voting step uses only the stable segments. The stable segmentation gives a coarse level of object/context division and reduces the computational complexity of context voting compared to the large number of finer object proposals, while still maintaining a semantically informative contextual inference. 

\subsection{Datasets}
We conduct our training and experiments on the Pascal VOC dataset.~\cite{Everingham10} which is a \textit{de facto} benchmark for object detection. Since the original dataset does not provide annotation of segmentation and contextual classes, we train our policy using the Pascal Context dataset~\cite{mottaghi2014role} which fully annotates every pixel of the Pascal VOC 2010 train and validation sets, with additional contextual classes such as sky, grass, ground, building etc., which is adequate for our purposes. We use the 33 context classes from~\cite{mottaghi2014role} and train our policy on the Pascal Context training set, and test our algorithm and baselines on the validation set. We also test our policy on the MSRC dataset~\cite{shotton2006textonboost} to show our algorithm can generalize to different data. 

\subsection{Feature Representation}
To classify object proposals, we extract region features and classify them using the deep neural network model in~\cite{BharathECCV2014} fine-tuned on Pascal VOC 2012. For the policy action classifiers, we also use the same model to extract features for states represented by the masks of search area $X^t$ and observed area $O^t$ in state $s_t$, then concatenate the features as inputs to the policy. For context classifiers we use a subset of the appearance features for superpixels from~\cite{tighe2010superparsing} and learn one-vs-all SVM models for classification.

\section{Experiments}

\subsection{Reduction of Number of Object Proposals}

Figure~\ref{fig:mapVSnumprop} shows that our 20 questions detection algorithm can effectively reduce a large amount of object proposals ($30\% \sim 40\%$) while maintaining similar mAP performance compared to exhaustive detection on all object proposals.  

\subsection{Comparison with other context based methods}

\subsection{Comparison with random search methods}


% ferrarri 2012
% ~\cite{bogdan2012context} 25000 to 100 

% \section{Conclusion}
In this paper, we proposed a sequential and dynamic process for the challenging problem of object detection and segmentation. In this process a policy iteratively selects a context related question adapting to different query and responses from previous questions, then more accurately refines the search area or rejects the object early without running many detectors, thus it can significantly reduce the computational cost while preserving good accuracy for the target. We formulate the object detection problem as a Markov Decision Process to learn a policy by imitation learning. We present a unified probabilistic framework to model spatial context between objects. We applied this active detection scheme to the problem of object detection and segmentation, and achieved comparable or even higher average precision with significant computational savings. 






{\small
\bibliographystyle{ieee}
\bibliography{nips2015}
}

\end{document}

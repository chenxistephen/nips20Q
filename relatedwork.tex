\section{Related Work}
\label{sec:relatedwork}

{\bf{Sequential Testing}}. 
The ``20 question'' approach to pattern recognition dates back to Blanchard and Geman~\cite{blanchard2005hierarchical}, motivated by the scene interpretation problem with a large number of possible explanations. They formally studied coarse-to-fine search in the theoretical framework of sequential hypothesis testing, and proposed optimal strategies considering both the cost and effectiveness of each test. Although they did not consider contextual information, their work provides a theoretical foundation for the design of sequential algorithms. ``20 questions" approaches recently have been used to generating questions for users in applications such as image binary segmentation~\cite{rupprecht2015image} and "visual Turing test''~\cite{geman2015visual}. But such methods involves human in the loop during test time, which is expensive and hard to scale up. 

There are several works~\cite{gao2011active} on objects classification by running classifiers sequentially in an active order.~\cite{branson2010visual} proposed an information gain based approach to iteratively pose questions for users and incorporate human responses and computer vision detector results for fine-grained classification.
~\cite{sergey2012timely} formulated object classification as a Markov decision process 
The model maintains a belief of object classes and keeps updating it based on new observations.  
However, these approaches only focus on classifying objects. They have not addressed the challenging problem of simultaneous segmentation and localization of objects in a multi-class scene as we do in this paper, and did not exploit inter-object spatial context.

~\cite{bogdan2012context} applied a sequential decision making framework to window selection. The next window is selected based on votes of previously evaluated windows. However, the voting process needs to look up nearest neighbors in hundreds of thousands of exemplar window pairs in the training set because their context is at the exemplar/instance level, which is highly inefficient. In contrast, our context modeling is semantically aware so we do not compute nearest neighbors over hundreds of thousands of windows in a high dimensional descriptor space to retrieve the voters, we only need votes from a few regions within the search space of context class instead of sampling hundreds of windows in~\cite{bogdan2012context}. Our context model achieves good accuracy while greatly reducing computational complexity.

{\bf Object Detection}. 
A common approach to object detection is based on applying gradient based features over densely sampled sliding windows~\cite{felzenszwalb2010object}.Such methods achieve good results on classes like human and vehicles, but they are very inefficient since they evaluate thousands of windows in an image, and false positve detections arise. To reduce the number of windows evaluated.~\cite{lampert2009efficient} proposed a subwindow search based on a branch-and-bound scheme and only evaluates the high scoring windows. Recently, category independent object proposals~\cite{carreira2012cpmc,van2011segmentation,arbelaez2014multiscale} have been proposed to generate a small number of high quality regions or windows that are likely to be objects. These approaches dramatically reduce the number of candidates and reduce false positive detections. Using these object proposals~\cite{girshick14CVPR, BharathECCV2014} train and apply deep neural network models on large datasets to learn the feature extractor and classifiers, and achieve state-of-the-art performance on the Pascal VOC detection challenge. 

{\bf Object Recognition using Context}. 
Context has been shown to improve object recognition and detection. Model-based approaches learn the appearance of semantic categories and relations among them
using a parametric model. In~\cite{gould2009decomposing, galleguillos2010context,mottaghirole, shotton2006textonboost, ladicky2010graph}, CRF models are used to combine unary potentials based on visual features extracted from superpixels with neighborhood constraints and low level context. Inter-object context in the scene has also been shown to improve recognition~\cite{galleguillos2010context, chen2011piecing}. Most of these context models are used as post-detection smoothing after all classifers are run as unary potentials, and then they are jointly incorporated in inference regardless of their importance to different kinds of objects and scenes. Our framework, in contrast, evaluates the informativeness of context in an active loop before classifications of all objects are made, and goes beyond simple co-occurence statistics.
